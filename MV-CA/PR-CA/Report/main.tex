\documentclass[12pt]{article}
\usepackage{tabularx} % extra features for tabular environment
\usepackage{amsmath}  % improve math presentation
\usepackage{graphicx} % takes care of graphic including machinery
\usepackage[margin=1in,letterpaper]{geometry} % decreases margins
\usepackage{cite} % takes care of citations
\usepackage[final]{hyperref} % adds hyper links inside the generated pdf file
\usepackage{leftidx}
\hypersetup{
	colorlinks=true,       % false: boxed links; true: colored links
	linkcolor=blue,        % color of internal links
	citecolor=blue,        % color of links to bibliography
	filecolor=magenta,     % color of file links
	urlcolor=blue
}
\usepackage{amsfonts}
\usepackage{enumerate}
\usepackage{paralist}
\usepackage{amssymb}
\usepackage{algorithm}
\usepackage{algorithmic}
\usepackage{verbatim}

\begin{document}

\title{Assignment Report of Machine Vision\\
	 \large Evaluation of RVFL Networks}
\author{Liu Xiangyu G1802061L}
\date{\today}
\maketitle

\section{Introduction}
Following the requirements of the assignment, the influences of the following various issues on the performance of RVFL Networks are evaluated:
\begin{enumerate}
	\item With and without direct links from the input layer to the output layer.
	\item With and without the bias in the output neuron.
	\item Effect of scaling the random features before feeding them into the activation function.
	\item Performances of radbas and hardlim activation functions.
	\item Performance of Moore-Penrose pseudoinverse and ridge regression for the computation of the output weights.
\end{enumerate}

\section{Datasets}
11 datasets are used in the evaluation, showed in Table \ref{table:datasets}.

\begin{table}[h]
\centering
\caption{Datasets used}
\begin{tabular}{llrrr}
	\hline
	Dataset\_id&Dataset\_name&Patterns&Features&Classes\\
	\hline
	1&Abalone&4177&8&3\\
	2&Car&1728&6&4\\
	3&Chess\_krvkp&3196&36&2\\
	4&Contrac&1473&9&3\\
	5&Magic&19020&10&2\\
	6&Molec\_biol\_splice&3190&60&3\\
	7&Monks\_3&3190&6&2\\
	8&Mushroom&8124&21&2\\
	9&Musk\_2&6598&166&2\\
	10&Nursery&12960&8&5\\
	11&Optical&3823&62&10\\
	\hline
\label{table:datasets}
\end{tabular}
\end{table}

\section{Evaluation Results}
\subsection{Evaluation Ranks}

\subsection{Direct Links}
The evaluation rank values considering 
\begin{inparaenum}[(i)]
	\item with or without direct links,
	\item with or without bias,
	\item using radbas or hardlim activation function,
\end{inparaenum}
  is showed in Table \ref{table:directlinkresult}.
\begin{table}[h]
\centering
\caption{Average Rank Values}
\begin{tabular}{llrrrr}
	\hline
	Computation Method&Act Func&-bias,-link&+bias,-link&-bias,+link&+bias,+link\\
	\hline
	Ridge Regression&Radbas&4177&8&3\\
	&Hardlim&6&4\\
	Moore-Penrose Pseudoinverse&Radbas&36&2\\
	&Hardlim&1473&9&3\\
	\hline
\label{table:directlinkresult}
\end{tabular}
\end{table}

\subsection{Bias}
\subsection{Radbas and Hardlim Activation Functions}
\subsection{Moore-Penrose Pseudoinverse and Ridge Regression}
\subsection{Scaling of Random Features}

\end{document}
